%%==================================================
%% thanks.tex for SJTU Master Thesis
%% based on CASthesis
%% modified by wei.jianwen@gmail.com
%% version: 0.3a
%% Encoding: UTF-8
%% last update: Dec 5th, 2010
%%==================================================

\begin{thanks}

在我攻读硕士学位的这不到三年的时间里,有很多老师、同学和朋友们给了我真诚的帮助和指导,才使我能够愉快地度过研究生阶段生活并顺利地完成学业。在本论文即将完成之际,谨向他们表达我最诚挚的感谢。

首先要感谢我的导师--朱浩瑾老师,在攻读硕士期间,他无论是在科研工作还是日常生活方面都给了我无微不至的指导和关心。在撰写论文的全过程,包括选题、资料搜集、文章结构组织、语言表达乃至很多细节的方面,他都对我进行了详细的指导。此外,他不但具有广播的学识和一丝不苟的科研态度,还有着令人敬仰的无限人格魅力,他对我的潜移默化的影响也将是我一生的财富。

同时我还要感谢学弟方晨廖晖,他凭借扎实的代码功底在我论文的实验部分给予了大量帮助。感谢NSEC实验室和所有实验室里的师兄师弟们,感谢他们陪伴我度过了人生中很美好的一段时光,也感谢他们对我平时工作生活中的无私的帮助。

最后我要感谢我的爱人,他对我一如既往的支持和鼓励一直是我努力工作的动力。

由于学术水平有限,我的论文难免有不足之处,恳请各位老师和学友批评和指正!
  

\end{thanks}
