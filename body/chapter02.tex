%%==================================================
%% chapter02.tex for SJTU Master Thesis
%% based on CASthesis
%% modified by wei.jianwen@gmail.com
%% Encoding: UTF-8
%%==================================================

\chapter{相关工作}\label{chap:related_work}

本文重点关注数据库驱动认知无线电网络中的位置隐私和轨迹隐私问题。认知无线电网络作为一种无线网络体制,面临着所有无线网络存在的安全和隐私问题。此外,数据库驱动认知无线电网络作为一种基于位置的服务,还将面临着传统基于位置服务系统中所固有的位置隐私和轨迹隐私泄露问题。本章将对相关研究领域的研究成果做一概述。数据库驱动认知无线电网络作为一种有应用前景的新兴认知无线电体制,其安全和隐私等问题必将受到越来越多的关注。

\section{数据库驱动认知无线电网络}\label{sec:database-driven}

数据库驱动认知无线电网络的概念提出时间较短,因此当前相关的文献较少,本小节对已有的关于数据库驱动认知无线电网络机制设计方面的文献进行简要描述。文献\cite{gurney2008geo}提出了一个基于地理信息数据库的认知无线电系统,认知用户可以从地理信息数据库中了解到网络中主用户的位置以及相关的环境信息,因此可以根据相关的传输模型计算可用频谱资源。文章指出,由于频谱感知需要设定一个安全的能量阈值,因此导致大量的频谱资源浪费。相比之下,基于地理信息数据库的认知系统可以更准确地获得可用频谱信息。文献\cite{murty2012senseless}提出通过数据库管理射频频谱资源并首次提出了白空间数据库驱动认知无线电网络Senseless,其中所有认知用户通过查询数据库获得广播电视频段的白空间,数据库对主用户和认知用户的状态和参数进行实时更新并基于Longley-Rice传输模型来预测频谱可用性。文献\cite{ying2013exploring}提出了一种室内的白空间数据库驱动认知无线电系统WISER,作者指出在室外和室内有分别超过$50\%$和$70\%$的白空间频谱。该系统在需要的位置上部署若干频谱感知传感器,由数据库收集数据并预测大楼内部的频谱可用情况从而提供给认知用户。文献\cite{zhangvehicle}指出现有的数据库驱动认知无线电网络模型由于传输模型的原因存在信号强度预测不准确的情况。因此作者提出了V-Scope,利用携带频谱感知传感器的公交车系统来收集沿途的无线频谱可用信息并构建频谱信息数据库。Senseless的优点是减轻了频谱测量的开销,但是由于在Longley-Rice模型中只考虑了地形因素,因此在有遮挡时无法精准地表达传输模型。此外,由于系统没有考虑认知用户的发射功率等因素,因此无法预测信道的质量;WISER通过大量传感器收集频谱数据,由于部署成本较高,只适合应用在较小的物理范围内,可扩展性不好。V-Scope集成了Senseless和WISER的优点,综合考虑了各种环境因素,通过广泛收集数据来重新提炼信号传输模型,并且部署容易。但缺点是测量和存储的开销较大。目前,FCC已经批准Google Inc.等厂家作为频谱数据库的运营者。在美国,无线广播电视频段的频谱可用信息已经对外公开公布,用户可以在诸如TVFool等网站查询。这意味着距离数据库驱动认知无线电网络在实际环境中部署又近了一步。

\section{隐私保护}\label{sec:location-privacy}

隐私是关于个人或机构等实体不愿意被外部揭露的信息,比如行为方式、爱好兴趣、健康情况等。一个实际应用系统的隐私保护水平是衡量其系统可行性的一个关键指标。隐私问题一直是学术界的研究重点,已有大量学者做了深入研究。文献\cite{wang2014location}对主流的隐私保护技术进行了总结并大致归为以下几类:匿名(空间隐形)、随机加扰、差分隐私、密码技术等。匿名的主要思路是将若干个用户分成组,然后对组内的用户进行混淆,打乱用户身份和敏感信息的对应关系。$k$-anonymity\cite{sweeney2002k}和$l$-diversity\cite{machanavajjhala2007diversity}是当前主流的两种匿名技术。$k$-anonymity将用户分成若干个混淆组并要求分组内个体数量不少于$k$,即分组内的用户与其他至少$k-1$个用户是不可区分的;$l$-diversity将用户的信息分为准标识符和敏感属性,要求匿名组内的个体被映射到某个敏感属性的概率不超过$1/l$。($\alpha,k$)-anonymity\cite{wong2006alpha}是一种改进的$k$-anonymity机制,综合考虑了$k$-anonymity和$l$-diversity的要求,它除了要求组内用户至少与其他$k-1$个用户不可区分之外,还要求敏感属性在组内出现的频率不超过$\alpha k$。
随机加扰\cite{adam1989security}是一种隐私保护的从个体中提取数据的方法,在诸如合作感知\cite{liu2012cloud}、数据挖掘\cite{du2003using}等许多隐私保护的应用中被广泛采用。其基本思路是用人造的随机数据取代原始的敏感数据,但同时需要保证替换后的数据的某些统计特性与原始数据相同,这种方法能够隐藏真实敏感数据,同时能够保证攻击者无法将隐私数据和用户个体关联起来。
差分隐私\cite{dwork2006differential}是当前比较流行的一种隐私模型,其基本思路是保证移除或增加一条数据记录不会很明显地影响输出结果。差分隐私定义了一个比较严格的攻击模型,对隐私泄露风险给出了定量的表示和证明。差分隐私保护基于数据失真技术,仅通过加入极少量的噪声就能够达到较高的隐私保护效果。差分隐私在学术界引起了较高的关注度。文献\cite{cormode2011differentially}提出针对特定的数据类型,通过简化步骤和降低敏感度等方式解决了稀疏数据在差分隐私保护过程中噪音添加量过大的问题。文献\cite{dwork2010differential}还基于差分隐私的思想提出了隐私保护性能更好的泛隐私概念。文献\cite{li2011provably}还首次提出将差分隐私与$k$-anonymity算法结合并用于解决微数据隐私保护的数据发布问题。
基于密码技术的隐私保护方法涵盖范围比较广,私有信息提取(Private Information Retrieval,PIR)\cite{chor1998private}就是在数据库查询过程中的一种高效的隐私保护方法。在一般的数据库查询过程中,私有信息提取使得用户能够在不暴露查询条目的前提下进行查询,即查询的内容对数据库来说是不可见的。


特别地,随着移动应用的发展,LBS的出现给人们生活带来了极大的便利,同时也使得用户的位置隐私面临严重的泄露风险。在位置隐私保护方面,上述几种通用的数据隐私保护方案或其变种一般都能够取得一定的位置隐私保护效果。此外,针对LBS中的位置隐私泄露问题,学术界已积累了大量的研究成果。文献\cite{freudiger2012evaluating}基于现实中的移动轨迹数据对LBS中的位置隐私进行了量化和评估,说明了LBS中的位置推断攻击不仅能够识别用户身份,可能还会导致更加严重的隐私泄露风险。文献\cite{kido2005anonymous}提出一种匿名通信的方案,用户向服务器提供真实位置的同时也提供一些虚假的地址使得服务器无法分辨用户的真实位置。\cite{mokbel2006new}提出在LBS系统中引入可信任的中间层,使得用户可以自行设置隐私保护的需求,然后在用户端通过空间匿名手段将真实地址转换成虚拟的空间范围,并由数据库端的查询处理器负责对空间范围以及查询信息进行处理,从而在通信过程中保护用户的位置隐私。文献\cite{ghinita2008private}指出了匿名方法存在的一些弊端并提出将私有信息提取的方法用于LBS中,在无需可信任第三方匿名器的前提下能够通过数据挖掘技术计算用户实际位置周边的邻居信息。该方法还能够抵御基于相关性的推测攻击。文献\cite{gedik2005location}提出一种可定制的$k$-anonymity方案,它使得移动用户可以自行指定能够忍受的最小的匿名程度和最大的时间空间分辨率,并且在一个可信任的服务器上进行身份移除和位置信息的时间空间隐形。文献\cite{xu2009feeling}提出一种方案,用户可以根据不同位置所具有的不同信息敏感程度进行有区别的隐私保护,并在智能手机上实现了这个系统。文献\cite{shokri2012protecting}提出一种通用的基于贝叶斯零和博弈的位置隐私攻击与保护框架,在此框架下,攻击者可以根据对用户的先验知识对其进行最优的攻击使得他估计的误差距离最小,同时用户也可以针对攻击者可能选择的最优策略来制定自己的最优策略使得攻击者获得的误差距离最大,从而最大程度上保护自己的位置隐私。
轨迹隐私不同于位置隐私,对轨迹数据进行挖掘可以获得更加丰富的信息,例如美国曾经利用GPS数据分析
交通设施建设中的问题,某些公司也可以通过分析雇员的上下班运动轨迹数据以提高员工的工作效率。一旦轨迹数据用于非法用途,则可能造成很严重的后果。在轨迹隐私保护方面,很多传统的保护位置隐私的方法却无法实现对轨迹隐私的保护。文献\cite{luper2007spatial}提出一种基于假数据的轨迹隐私保护技术,通过添加假的轨迹数据对原始轨迹数据进行干扰,同时又保证了被干扰的轨迹数据的统计特性属性不发生严重失真。文献\cite{abul2008never}提出基于泛化法的轨迹隐私保护技术,将运动轨迹上所有的采样点都泛化为对应的匿名区域以实现隐私保护。文献\cite{terrovitis2008privacy}提出一种基于抑制的轨迹隐私保护方案,可根据具体情况有条件地发布轨迹数据从而实现敏感数据的隐私保护。

\section{认知无线电网络中的安全与隐私}

认知无线电网络具有传统无线网络所具备的所有安全问题,如无线信号的截获和篡改等。此外,随着“认知”功能的引入,还带来很多新的安全隐患。目前已有不少文献提出了认知无线电网络中潜在的安全威胁和应对策略。认知无线电网络存在的主要安全问题有模仿主用户攻击\cite{chen2008defense}、控制信道干扰\cite{bian2006mac}、拒绝服务攻击、消息窃听、GPS干扰等。在认知无线电网络的隐私问题上,文献\cite{li2012location}基于真实环境数据实验指出合作频谱感知过程中,恶意的服务提供者或者认知用户可以通过对频谱感知数据的分析来对网络中的认知用户进行定位。如果数据融合过程不采用一定的隐私保护措施,那么单个用户的频谱感知结果可以从融合结果中提取出来,进而导致单个用户的位置隐私泄露。而隐私泄露风险可能会使用户不愿意去参与频谱感知进而影响网络整体性能。
之前的大部分频谱感知方面的研究都把改善频谱感知性能\cite{chen2008robust,quan2009optimal}和其它安全问题\cite{he2013byzantine,li2011believe}作为重点,而对用户隐私考虑得不多。文献\cite{li2012location}提出基于单个感知报告的位置隐私推断和基于差分隐私的位置推断攻击,并提出了基于密码技术的感知结果融合方案来阻止合作频谱感知中用户上报的感知数据被非法获取。
在数据库驱动认知无线电网络概念提出后,作为一种LBS应用,其安全和隐私问题受到了一定关注。文献
\cite{gao2013location}提出一种用于数据库驱动认知无线电网络中的基于私有信息提取的位置隐私保护技术,实现了在频谱查询过程中认知用户真实位置信息的隐藏。同时文章还指出,即便隐藏了认知用户的真实位置信息,数据库还是能够根据其使用过的信道来实施位置推断攻击从而破坏认知用户的位置隐私。文献\cite{bahrak2014protecting}指出了数据库驱动认知无线电网络中的频谱可用性信息与主用户位置存在很强的相关性,因此可基于该相关性对主用户的位置进行推断。

